% Exam Template for UMTYMP and Math Department courses
%
% Using Philip Hirschhorn's exam.cls: http://www-math.mit.edu/~psh/#ExamCls
%
% run pdflatex on a finished exam at least three times to do the grading table on front page.
%
%%%%%%%%%%%%%%%%%%%%%%%%%%%%%%%%%%%%%%%%%%%%%%%%%%%%%%%%%%%%%%%%%%%%%%%%%%%%%%%%%%%%%%%%%%%%%%

% These lines can probably stay unchanged, although you can remove the last
% two packages if you're not making pictures with tikz.
\documentclass[11pt]{exam}
\RequirePackage{amssymb, amsfonts, amsmath, latexsym, verbatim, xspace, setspace}
\RequirePackage{tikz, pgflibraryplotmarks}

% By default LaTeX uses large margins.  This doesn't work well on exams; problems
% end up in the "middle" of the page, reducing the amount of space for students
% to work on them.
\usepackage[margin=1in]{geometry}
\usepackage{enumerate}
\usepackage{amsthm}

\theoremstyle{definition}
\newtheorem{soln}{Solution}

% Here's where you edit the Class, Exam, Date, etc.
\newcommand{\class}{Math 350 Section 1}
\newcommand{\term}{Fall 2023}
\newcommand{\examnum}{Exam III}
\newcommand{\examdate}{November 15, 2023}
\newcommand{\timelimit}{1 Hour 50 Minutes}
\newcommand{\ol}[1]{\overline{#1}}

% For an exam, single spacing is most appropriate
\singlespacing
% \onehalfspacing
% \doublespacing

% For an exam, we generally want to turn off paragraph indentation
\parindent 0ex

\begin{document} 

% These commands set up the running header on the top of the exam pages
\pagestyle{head}
\firstpageheader{}{}{}
\runningheader{\class}{\examnum\ - Page \thepage\ of \numpages}{\examdate}
\runningheadrule

\begin{flushright}
\begin{tabular}{p{2.8in} r l}
\textbf{\class} & \textbf{Name (Print):} & \makebox[2in]{\hrulefill}\\
\textbf{\term} &&\\
\textbf{\examnum} & \textbf{Student ID:}&\makebox[2in]{\hrulefill}\\
\textbf{\examdate} &&\\
\textbf{Time Limit: \timelimit} % & Teaching Assistant & \makebox[2in]{\hrulefill}
\end{tabular}\\
\end{flushright}
\rule[1ex]{\textwidth}{.1pt}


This exam contains \numpages\ pages (including this cover page) and
\numquestions\ problems.  Check to see if any pages are missing.  Enter
all requested information on the top of this page, and put your initials
on the top of every page, in case the pages become separated.\\

You may \textit{not} use your books or notes on this exam.

You are required to show your work on each problem on this exam.  The following rules apply:\\

\begin{minipage}[t]{3.7in}
\vspace{0pt}
\begin{itemize}

%\item \textbf{If you use a ``fundamental theorem'' you must indicate this} and explain
%why the theorem may be applied.

\item \textbf{Organize your work}, in a reasonably neat and coherent way, in
the space provided. Work scattered all over the page without a clear ordering will 
receive very little credit.  

\item \textbf{Mysterious or unsupported answers will not receive full
credit}.  A correct answer, unsupported by calculations, explanation,
or algebraic work will receive no credit; an incorrect answer supported
by substantially correct calculations and explanations might still receive
partial credit.  This especially applies to limit calculations.

\item If you need more space, use the back of the pages; clearly indicate when you have done this.

\item \textbf{Box Your Answer} where appropriate, in order to clearly indicate what you consider the answer to the question to be.
\end{itemize}

Do not write in the table to the right.
\end{minipage}
\hfill
\begin{minipage}[t]{2.3in}
\vspace{0pt}
%\cellwidth{3em}
\gradetablestretch{2}
\vqword{Problem}
\addpoints % required here by exam.cls, even though questions haven't started yet.	
\gradetable[v]%[pages]  % Use [pages] to have grading table by page instead of question

\end{minipage}
\newpage % End of cover page

%%%%%%%%%%%%%%%%%%%%%%%%%%%%%%%%%%%%%%%%%%%%%%%%%%%%%%%%%%%%%%%%%%%%%%%%%%%%%%%%%%%%%
%
% See http://www-math.mit.edu/~psh/#ExamCls for full documentation, but the questions
% below give an idea of how to write questions [with parts] and have the points
% tracked automatically on the cover page.
%

%%%%%%%%%%%%%%%%%%%%%%%%%%%%%%%%%%%%%%%%%%%%%%%%%%%%%%%%%%%%%%%%%%%%%%%%%%%%%%%%%%%%%

\begin{questions}

\addpoints

\question[10]\mbox{}
\textbf{TRUE or FALSE!}  Write  TRUE if the statement is true.  Otherwise, write FALSE.  Your response should be in ALL CAPS.  No justification is required.
\begin{enumerate}[(a)]
\item  
There exists a function which is discontinuous at every real number
\vspace{1.3in}
\item
The radius of convergence of $\sum_{n=0}^\infty \frac{x^n}{n!}$ is $R=\infty$
\vspace{1.3in}
\item
If $f(x)^2$ is continuous at $x=a$, then $f(x)$ is continuous at $x=a$
\vspace{1.3in}
\item
The function $f(x) = \frac{1}{x^2+1}$ is continuous at every real number
\vspace{1.3in}
\item  
If $\lim_{x\rightarrow a+} f(x)$ and $\lim_{x\rightarrow a-}f(x)$ exist, then $\lim_{x\rightarrow a}f(x)$ exists
\vspace{1.3in}
\end{enumerate}

\newpage
\question[10]\mbox{}

For each of the following, either give an example or explain why an example does not exist.

\begin{enumerate}[(a)]
\item  
A function which is continuous at $x=0$ but NOT differentiable at $x=0$
\vspace{1.3in}
\item
A power series whose radius of convergence is $R=3$
\vspace{1.3in}
\item
A function which is simultaneously discontinuous at every rational number and continuous at every irrational number.
\vspace{1.3in}
\item
A function which is continuous on $[0,1]$ but NOT uniformly continuous on $[0,1]$.
\vspace{1.3in}
\item  
A sequence of polynomials which converge pointwise to $f(x) = \frac{1}{1-x}$ on $(-1,1)$.
\vspace{1.3in}
\end{enumerate}

\newpage
\question[10]\mbox{}

Consider the function
$$f(x) = \left\lbrace\begin{array}{cc}x, & x\in \mathbb{Q}\\0 & x\notin\mathbb{Q}\end{array}\right.$$

\begin{enumerate}[(a)]
\item Write down the $\epsilon$,$\delta$-definition of $\lim_{x\rightarrow a} f(x)$
\vspace{1in}
\item Let $a\in\mathbb{R}$ with $a\neq 0$.  Prove that $\lim_{x\rightarrow a}f(x)$ does not exist
\vspace{3in}
\item Prove that $\lim_{x\rightarrow 0} f(x)$ exists and is equal to $0$
\end{enumerate}

\newpage
\question[10]\mbox{}

For each of the following power series, determine the radius of convergence.  Show your work.

\begin{enumerate}[(a)]
\item $\sum_{n=0}^\infty n^nx^n$
\vspace{2in}
\item $\sum_{n=0}^\infty n^3x^n$
\vspace{2in}
\item $\sum_{n=0}^\infty 4^nx^{2n}$
\vspace{2in}
\item $\sum_{n=0}^\infty \left(1+\frac{1}{n}\right)^{2n}x^n$
\end{enumerate}

\newpage
\question[10]\mbox{}
Consider the sequence of functions $$f_n(x) = e^{-nx^2}$$ and the function

$$f(x) = \left\lbrace\begin{array}{cc}1 & x=0\\0 & x\neq 0\end{array}\right.$$

\begin{enumerate}[(a)]
\item Write down the definition of a sequence of functions $(f_n(x))$ converging uniformly to a function $f(x)$ on a set $S$
\vspace{1.5in}
\item Prove that the sequence of functions above converges pointwise to $f(x)$.
\vspace{3in}
\item Prove that the sequence of functions above does NOT converge uniformly to $f(x)$.  If you use a theorem from class, make sure to carefully state it.
\end{enumerate}


\newpage
\question[10]\mbox{}

\begin{enumerate}[(a)]
\item Write down the $\epsilon$,$\delta$-definition of continuity.
\vspace{2in}
\item Suppose that $f(x)$ is a continuous function on $\mathbb{R}$ with the property that $f(x) = f(x/2)$ for all real numbers $x$.  Prove that $f(x)$ must be a constant function.
\vspace{3in}
\item Show by example that (b) is false if we drop the continuity assumption.
\end{enumerate}

\newpage
\question[10]\mbox{}

\begin{enumerate}[(a)]
\item State the definition of $f(x)$ being uniformly continuous on a subset $S\subseteq\text{Dom}(f)$
\vspace{1.5in}
\item Using $\epsilon$ and $\delta$, prove that the function $f(x) = \frac{1}{x^2}$ is uniformly continuous on the interval $[2,\infty)$.
\vspace{3in}
\item Explain why $f(x)$ from part (b) is NOT uniformly continuous on $(0,\infty)$.
\end{enumerate}


\end{questions}

\end{document}

