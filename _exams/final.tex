% Exam Template for UMTYMP and Math Department courses
%
% Using Philip Hirschhorn's exam.cls: http://www-math.mit.edu/~psh/#ExamCls
%
% run pdflatex on a finished exam at least three times to do the grading table on front page.
%
%%%%%%%%%%%%%%%%%%%%%%%%%%%%%%%%%%%%%%%%%%%%%%%%%%%%%%%%%%%%%%%%%%%%%%%%%%%%%%%%%%%%%%%%%%%%%%

% These lines can probably stay unchanged, although you can remove the last
% two packages if you're not making pictures with tikz.
\documentclass[11pt]{exam}
\RequirePackage{amssymb, amsfonts, amsmath, latexsym, verbatim, xspace, setspace}
\RequirePackage{tikz, pgflibraryplotmarks}

% By default LaTeX uses large margins.  This doesn't work well on exams; problems
% end up in the "middle" of the page, reducing the amount of space for students
% to work on them.
\usepackage[margin=1in]{geometry}
\usepackage{enumerate}
\usepackage{amsthm}

\theoremstyle{definition}
\newtheorem{soln}{Solution}

% Here's where you edit the Class, Exam, Date, etc.
\newcommand{\class}{Math 350 Section 1}
\newcommand{\term}{Fall 2023}
\newcommand{\examnum}{Final Exam}
\newcommand{\examdate}{December 10, 2023}
\newcommand{\timelimit}{1 Hour 50 Minutes}
\newcommand{\ol}[1]{\overline{#1}}

% For an exam, single spacing is most appropriate
\singlespacing
% \onehalfspacing
% \doublespacing

% For an exam, we generally want to turn off paragraph indentation
\parindent 0ex

\begin{document} 

% These commands set up the running header on the top of the exam pages
\pagestyle{head}
\firstpageheader{}{}{}
\runningheader{\class}{\examnum\ - Page \thepage\ of \numpages}{\examdate}
\runningheadrule

\begin{flushright}
\begin{tabular}{p{2.8in} r l}
\textbf{\class} & \textbf{Name (Print):} & \makebox[2in]{\hrulefill}\\
\textbf{\term} &&\\
\textbf{\examnum} & \textbf{Student ID:}&\makebox[2in]{\hrulefill}\\
\textbf{\examdate} &&\\
\textbf{Time Limit: \timelimit} % & Teaching Assistant & \makebox[2in]{\hrulefill}
\end{tabular}\\
\end{flushright}
\rule[1ex]{\textwidth}{.1pt}


This exam contains \numpages\ pages (including this cover page) and
\numquestions\ problems.  Check to see if any pages are missing.  Enter
all requested information on the top of this page, and put your initials
on the top of every page, in case the pages become separated.\\

You may \textit{not} use your books or notes on this exam.

You are required to show your work on each problem on this exam.  The following rules apply:\\

\begin{minipage}[t]{3.7in}
\vspace{0pt}
\begin{itemize}

%\item \textbf{If you use a ``fundamental theorem'' you must indicate this} and explain
%why the theorem may be applied.

\item \textbf{Organize your work}, in a reasonably neat and coherent way, in
the space provided. Work scattered all over the page without a clear ordering will 
receive very little credit.  

\item \textbf{Mysterious or unsupported answers will not receive full
credit}.  A correct answer, unsupported by calculations, explanation,
or algebraic work will receive no credit; an incorrect answer supported
by substantially correct calculations and explanations might still receive
partial credit.  This especially applies to limit calculations.

\item If you need more space, use the back of the pages; clearly indicate when you have done this.

\item \textbf{Box Your Answer} where appropriate, in order to clearly indicate what you consider the answer to the question to be.
\end{itemize}

Do not write in the table to the right.
\end{minipage}
\hfill
\begin{minipage}[t]{2.3in}
\vspace{0pt}
%\cellwidth{3em}
\gradetablestretch{2}
\vqword{Problem}
\addpoints % required here by exam.cls, even though questions haven't started yet.	
\gradetable[v]%[pages]  % Use [pages] to have grading table by page instead of question

\end{minipage}
\newpage % End of cover page

%%%%%%%%%%%%%%%%%%%%%%%%%%%%%%%%%%%%%%%%%%%%%%%%%%%%%%%%%%%%%%%%%%%%%%%%%%%%%%%%%%%%%
%
% See http://www-math.mit.edu/~psh/#ExamCls for full documentation, but the questions
% below give an idea of how to write questions [with parts] and have the points
% tracked automatically on the cover page.
%

%%%%%%%%%%%%%%%%%%%%%%%%%%%%%%%%%%%%%%%%%%%%%%%%%%%%%%%%%%%%%%%%%%%%%%%%%%%%%%%%%%%%%

\begin{questions}

\addpoints

\question[10]\mbox{}
\textbf{TRUE or FALSE!}  Write  TRUE if the statement is true.  Otherwise, write FALSE.  Your response should be in ALL CAPS.  No justification is required.
\begin{enumerate}[(a)]
\item  
If $f(x)/x$ is continuous at $x=1$, then $f(x)$ is continuous at $x=1$
\vspace{1.3in}
\item
If the sequential limit $\lim_{n\rightarrow\infty} f(1/n)$ is equal to $f(0)$, then $f(x)$ is continuous at $x=0$
\vspace{1.3in}
\item
If $f(x)$ is uniformly continuous on $(0,1)$, then $f(x)$ extends to a continuous function on $[0,1]$
\vspace{1.3in}
\item
If $f(x)$ is a monotone increasing function on $[a,b]$, then $f(x)$ is Darboux integrable on $[a,b]$
\vspace{1.3in}
\item  If the domain of $f(x)$ is $\mathbb{R}$ and the range of $f(x)$ is $\{y\in\mathbb{R}: \lvert y\rvert > 1\}$, then $f(x)$ has a discontinuity
\vspace{1.3in}
\end{enumerate}

\newpage
\question[10]\mbox{}

For each of the following, either give an example or explain why an example does not exist.

\begin{enumerate}[(a)]
\item  
A bounded sequence with no convergent subsequence
\vspace{1.3in}
\item
A function which is not Riemann integrable
\vspace{1.3in}
\item
A power series whose radius of convergence is $R=0$
\vspace{1.3in}
\item
A function $f(x)$ differentiable everywhere with $f(1)-f(0) > f'(x)$ for all $x\in (0,1)$
\vspace{1.3in}
\item
A sequence of functions differentiable at $x=0$ which converge uniformly on $[-1,1]$ to a function which is not differentiable at $x=0$
\vspace{1.3in}
\end{enumerate}

\newpage
\question[10]\mbox{}
Let $(a_n)$ and $(b_n)$ be bounded sequences.
\begin{enumerate}[(a)]
\item Define $\limsup a_n$ and $\liminf a_n$
\vspace{2in}
\item Prove

$$\limsup (a_n+b_n)\leq \limsup a_n + \limsup b_n$$
\vspace{3in}

\item

Give an explicit example of two sequences $(a_n)$ and $(b_n)$ for which the above inequality is strict (not equal).
\end{enumerate}


\newpage
\question[10]\mbox{}

\begin{enumerate}[(a)]
\item Write down the $\epsilon$,$\delta$-definition of continuity.
\vspace{2in}
\item Suppose that $f(x)$ and $g(x)$ are continuous functions on $\mathbb{R}$ with the property that $f(x) = g(x)$ for all rational numbers $x$.  Prove that $f(x)=g(x)$ must be true for all real numbers $x$.
\vspace{3in}
\item Show by example that (b) is false if we drop the continuity assumption.
\end{enumerate}


\newpage
\question[10]\mbox{}

\begin{enumerate}[(a)]
\item Write down the definition of $f(x)$ being differentiable at $x=a$.
\vspace{1.3in}
\item Prove using only basic definitions and algebra that $f(x) = x^{1/3}$ is differentiable a $x=8$.
\vspace{4in}
\item Carefully prove that $f(x) = x^{1/3}$ is not differentiable at $x=0$
\end{enumerate}

\newpage
\question[10]\mbox{}

\begin{enumerate}[(a)]
\item State what it means for $f(x)$ to be uniformly continuous on a subset $S\subseteq\text{Dom}(f)$
\vspace{1.3in}
\item State the Mean Value Theorem
\vspace{1.3in}
\item Suppose that $f(x)$ is differentiable on $\mathbb R$ and that there exists a constant $M>0$ such that $\lvert f'(x)\rvert < M$ for all $x$.  Prove that $f(x)$ must be uniformly continuous on $\mathbb{R}$
\end{enumerate}


\newpage
\question[10]\mbox{}

Consider the function
$$f(x) = \left\lbrace\begin{array}{cc}5, & x\in \mathbb{Q}\\0 & x\notin\mathbb{Q}\end{array}\right.$$

\begin{enumerate}[(a)]
\item  Write the definition of the upper Darboux sum $U(f,P)$ of $f(x)$ on a partition $P = \{a_0,\dots, a_n\}$ on $[a,b]$
\vspace{1.5in}
\item Write the definition of the lower Darboux integral $L(f)$ of $f(x)$ on $[a,b]$
\vspace{1.5in}
\item 
Evaluate $L(f)$ and $U(f)$.
\end{enumerate}


\end{questions}

\end{document}

