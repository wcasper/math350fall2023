% Exam Template for UMTYMP and Math Department courses
%
% Using Philip Hirschhorn's exam.cls: http://www-math.mit.edu/~psh/#ExamCls
%
% run pdflatex on a finished exam at least three times to do the grading table on front page.
%
%%%%%%%%%%%%%%%%%%%%%%%%%%%%%%%%%%%%%%%%%%%%%%%%%%%%%%%%%%%%%%%%%%%%%%%%%%%%%%%%%%%%%%%%%%%%%%

% These lines can probably stay unchanged, although you can remove the last
% two packages if you're not making pictures with tikz.
\documentclass[11pt]{exam}
\RequirePackage{amssymb, amsfonts, amsmath, latexsym, verbatim, xspace, setspace}
\RequirePackage{tikz, pgflibraryplotmarks}

% By default LaTeX uses large margins.  This doesn't work well on exams; problems
% end up in the "middle" of the page, reducing the amount of space for students
% to work on them.
\usepackage[margin=1in]{geometry}
\usepackage{enumerate}
\usepackage{amsthm}

\theoremstyle{definition}
\newtheorem{soln}{Solution}

% Here's where you edit the Class, Exam, Date, etc.
\newcommand{\class}{Math 350 Section 1}
\newcommand{\term}{Fall 2023}
\newcommand{\examnum}{Exam II}
\newcommand{\examdate}{October 17, 2023}
\newcommand{\timelimit}{1 Hour 50 Minutes}
\newcommand{\ol}[1]{\overline{#1}}

% For an exam, single spacing is most appropriate
\singlespacing
% \onehalfspacing
% \doublespacing

% For an exam, we generally want to turn off paragraph indentation
\parindent 0ex

\begin{document} 

% These commands set up the running header on the top of the exam pages
\pagestyle{head}
\firstpageheader{}{}{}
\runningheader{\class}{\examnum\ - Page \thepage\ of \numpages}{\examdate}
\runningheadrule

\begin{flushright}
\begin{tabular}{p{2.8in} r l}
\textbf{\class} & \textbf{Name (Print):} & \makebox[2in]{\hrulefill}\\
\textbf{\term} &&\\
\textbf{\examnum} & \textbf{Student ID:}&\makebox[2in]{\hrulefill}\\
\textbf{\examdate} &&\\
\textbf{Time Limit: \timelimit} % & Teaching Assistant & \makebox[2in]{\hrulefill}
\end{tabular}\\
\end{flushright}
\rule[1ex]{\textwidth}{.1pt}


This exam contains \numpages\ pages (including this cover page) and
\numquestions\ problems.  Check to see if any pages are missing.  Enter
all requested information on the top of this page, and put your initials
on the top of every page, in case the pages become separated.\\

You may \textit{not} use your books or notes on this exam.

You are required to show your work on each problem on this exam.  The following rules apply:\\

\begin{minipage}[t]{3.7in}
\vspace{0pt}
\begin{itemize}

%\item \textbf{If you use a ``fundamental theorem'' you must indicate this} and explain
%why the theorem may be applied.

\item \textbf{Organize your work}, in a reasonably neat and coherent way, in
the space provided. Work scattered all over the page without a clear ordering will 
receive very little credit.  

\item \textbf{Mysterious or unsupported answers will not receive full
credit}.  A correct answer, unsupported by calculations, explanation,
or algebraic work will receive no credit; an incorrect answer supported
by substantially correct calculations and explanations might still receive
partial credit.  This especially applies to limit calculations.

\item If you need more space, use the back of the pages; clearly indicate when you have done this.

\item \textbf{Box Your Answer} where appropriate, in order to clearly indicate what you consider the answer to the question to be.
\end{itemize}

Do not write in the table to the right.
\end{minipage}
\hfill
\begin{minipage}[t]{2.3in}
\vspace{0pt}
%\cellwidth{3em}
\gradetablestretch{2}
\vqword{Problem}
\addpoints % required here by exam.cls, even though questions haven't started yet.	
\gradetable[v]%[pages]  % Use [pages] to have grading table by page instead of question

\end{minipage}
\newpage % End of cover page

%%%%%%%%%%%%%%%%%%%%%%%%%%%%%%%%%%%%%%%%%%%%%%%%%%%%%%%%%%%%%%%%%%%%%%%%%%%%%%%%%%%%%
%
% See http://www-math.mit.edu/~psh/#ExamCls for full documentation, but the questions
% below give an idea of how to write questions [with parts] and have the points
% tracked automatically on the cover page.
%

%%%%%%%%%%%%%%%%%%%%%%%%%%%%%%%%%%%%%%%%%%%%%%%%%%%%%%%%%%%%%%%%%%%%%%%%%%%%%%%%%%%%%

\begin{questions}

\addpoints

\question[10]\mbox{}
\textbf{TRUE or FALSE!}  Write  TRUE if the statement is true.  Otherwise, write FALSE.  Your response should be in ALL CAPS.  No justification is required.
\begin{enumerate}[(a)]
\item  
If $\lim a_n = 0$ then $\sum_{n=1}^\infty a_n$ converges
\vspace{1.3in}
\item
If $\limsup a_n=\liminf a_n$ then $a_n$ is the constant sequence
\vspace{1.3in}
\item
A bounded sequence must have a Cauchy subsequence
\vspace{1.3in}
\item
An absolutely convergent series must be convergent
\vspace{1.3in}
\item  
There exists a sequence $(s_n)$ where the value of every real number appears at least once
\vspace{1.3in}
\end{enumerate}

\newpage
\question[10]\mbox{}

Let $(a_n)$ and $(b_n)$ be bounded sequences.
\begin{enumerate}[(a)]
\item Prove

$$\limsup (a_n+b_n)\leq \limsup a_n + \limsup b_n$$
\vspace{2.5in}
\item Prove

$$\limsup a_n + \liminf b_n \leq\limsup (a_n+b_n)$$
\vspace{2.5in}

\item

Give an explicit example of two sequences $(a_n)$ and $(b_n)$ for which 

$$\limsup (a_n+b_n)\neq \limsup a_n + \limsup b_n.$$

\end{enumerate}


\newpage
\question[10]\mbox{}

Let $r\in\mathbb{R}$.
\begin{enumerate}[(a)]
\item Write down a closed form expression for the sum

$$\sum_{k=0}^{n-1} r^k = 1 + r + r^2 + \dots + r^{n-1}.$$

\vspace{1in}
\item Use (a) to prove that if $|r| < 1$ then

$$\sum_{k=0}^\infty r^k = \frac{1}{1-r}.$$
\vspace{2in}

\item Find an exact fractional expression for

$$0.2023202320232023\dots = 0.\overline{2023}$$

\end{enumerate}

\newpage
\question[10]\mbox{}

\begin{enumerate}[(a)]
\item  Write down the definition of $\sum_{n=1}^\infty a_n$ converging.
\vspace{1in}
\item  Let $k\in\mathbb R$ and suppose that $\sum_{n=1}^\infty a_n$ converges.  Prove that $\sum_{n=1}^\infty ka_n$ converges.
\vspace{2in}
\item  Suppose $\sum_{n=1}^\infty a_n$ and $\sum_{n=1}^\infty b_n$ converge.  Prove $\sum_{n=1}^\infty (a_n+b_n)$ converges.
\vspace{2in}
\item  Suppose $\sum_{n=1}^\infty a_n$ and $\sum_{n=1}^\infty b_n$ are absolutely convergent.  Prove $\sum_{n=1}^\infty a_nb_n$ converges.
\end{enumerate}

\newpage
\question[10]\mbox{}

For each of the following series, determine if it diverges, converges, or converges absolutely.
Carefully justify your answer.

\begin{enumerate}[(a)]
\item  
$$\sum_{n=1}^\infty \frac{n}{n^3+1}$$
\vspace{0.9in}
\item
$$\sum_{n=1}^\infty \frac{2^n}{n!}$$
\vspace{0.9in}
\item
$$\sum_{n=1}^\infty \frac{5^n}{n^n}$$
\vspace{0.9in}
\item
$$\sum_{n=1}^\infty \frac{\cos(n\pi)}{n}$$
\vspace{0.9in}
\item
$$\sum_{n=1}^\infty \frac{n!}{n^n}$$
\end{enumerate}

\newpage
\question[10]\mbox{}

Let $\mathbb{I}$ be the set of all irrational numbers in the interval $[0,1]$.
Consider the function

$$f: I\rightarrow [0,1]$$

defined via decimal expansions by

$$f(0.d_1d_2d_3d_4d_5d_6\dots) = 0.d_1d_3d_5d_7\dots$$

\begin{enumerate}[(a)]
\item Prove that $f(x)$ is not injective.
\vspace{2.5in}
\item Prove that $f(x)$ is surjective.
\vspace{2.5in}
\item If we replace $I$ with the interval $[0,1]$ above, prove that the function is no longer well-defined.  Carefully explain.
\end{enumerate}

\newpage
\question[10]\mbox{}


\begin{enumerate}[(a)]
\item Write down the $\epsilon$,$\delta$-definition of continuity.
\vspace{1in}
\item Consider the function
$$f(x) = \left\lbrace\begin{array}{cc}x, & x\in \mathbb{Q}\\0 & x\notin\mathbb{Q}\end{array}\right.$$
Let $a\in\mathbb{R}$ with $a\neq 0$.  Prove that $f(x)$ is not continuous at $x=a$.
\vspace{3in}
\item Prove that $f(x)$ is continuous at $x=0$.
\end{enumerate}

\end{questions}

\end{document}

